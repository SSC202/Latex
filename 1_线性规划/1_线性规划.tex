\documentclass{article}

\usepackage[UTF8]{ctex}
\usepackage{algorithm}
\usepackage{minted}
\usepackage{amsmath}
\usepackage{bm}
\usepackage[a4paper,left=10mm,right=10mm,top=15mm,bottom=15mm]{geometry}

\title{第一章 \quad 线性规划}

\begin{document}
\maketitle
\begin{abstract}
	本章阐述了线性规划的数学基本原理,例子和Python代码实现。
\end{abstract}

\section{数学原理}
线性规划模型的一般形式为:
	\begin{equation}
		\begin{split}
			max \quad z = c_1x_1 + c_2x_2 + \cdots + c_nx_n \\
			s.t.\begin{cases}
				a_{11}x_1 + a_{12}x_2 + \cdots + a_{1n}x_n \geq b_1 \\
				a_{21}x_1 + a_{22}x_2 + \cdots + a_{2n}x_n \geq b_2 \\
				\cdots											    \\
				a_{m1}x_1 + a_{n2}x_2 + \cdots + a_{nn}x_n \geq b_m	\\
				x_1,x_2,\cdots,x_n \geq 0							\\
 			\end{cases}
		\end{split}
	\end{equation}
其矩阵表示形式为:
	\begin{equation}
		\begin{split}
			max \quad z = \bm{c^Tx} \\
			s.t.\begin{cases}
				\bm{Ax} \geq \bm{b} \\
				\bm{x} \geq 0 		\\
			\end{cases}
		\end{split}
	\end{equation}
式中,$\bm{c^T}$为目标函数的系数向量,又称价值向量。$\bm{x}$称为决策向量,由决策者决定。$\bm{A}$为约束方程组的系数矩阵;$\bm{A}$的列向量为约束方程组的系数向量,$\bm{b}$称为约束方程组的常数向量。
\subsection{线性规划问题的解}
\pmb{可行解}:满足约束条件(2)的解$\bm{x} = [x_1,x_2,\cdots,x_n]^T$称为线性规划问题的可行解,其中使得目标函数$z$取得最大值的解称为最优解。

\pmb{可行域}:所有可行解构成的集合称为可行域。
\subsection{灵敏度分析}
线性规划问题中,$\bm{A}$,$\bm{b}$,$\bm{c}$不一定为常矩阵。其中的一些值变化时,可行解会发生一定的变化。

将产生以下问题:

(1)在条件变化时,现行的最优解会发生什么变化; 

(2)将参数变化限制在哪个范围内,原最优解仍是最优解。

对于数学建模问题而言,灵敏度分析是必要的。

\section{Python程序}

使用Python的spicy库可以解决线性规划问题:

对于Python而言,线性规划的标准形式如下:
	\begin{equation}
		\begin{split}
			max \quad \bm{c^Tx} \\
			s.t.\begin{cases}
				\bm{Ax} \leq \bm{b} \\ 
				\bm{A}_{eq}\bm{x} = \bm{b}_{eq} \\
				\bm{lb} \leq \bm{x} \leq \bm{\mu b}	\\
			\end{cases}
		\end{split}
	\end{equation}
对应的,Python的函数如下
\begin{minted}{python}
"""
	线性规划函数:
	@param c 价值向量
	@param A_ub 不等式约束矩阵
	@param b_ub 不等式约束向量
	@param A_eq 等式约束矩阵
	@param b_eq 等式约束向量
	@param bounds 列表元素的元组 定义决策变量的最小值

	@retval x 目标函数最小化的决策变量值
	@retval fun 目标函数最优值
	@retval slack 不等式约束的松弛值,理论上为正值
	@retval con 等式约束的残差,理论上为0
	@retval status 算法退出状态的整数
	@retval nit 迭代总数
"""
	scipy.optimize.lineprog(c,A_ub=None,b_ub=None,A_eq=None,b_eq=None,bounds=None,method='simplex',callback=None,options=None)
\end{minted}

\pmb{注1}:函数是默认求最小值的,如果要求最大值,应该将$\bm{c}$改为$\bm{-c}$。

\pmb{注2}:对于带绝对值的线性规划,先尝试使用拆分等手段进行线性化,然后进行线性规划。

\end{document}

